\documentclass[]{article}
\usepackage{graphicx}
\linespread{1}
\usepackage{amsmath}
\usepackage{siunitx}
\usepackage{tikz}
\usepackage[]{geometry}
\begin{document}

\section*{Helmholtz Coils for MRP}

The length of the saturated absorption cell is $l_c = \SI{8}{\centi \meter}$ and has a radius of $r_c = \SI{1.25}{\centi \meter}$. In order to create a magnetic field around this, we will use Helmholtz coils. The magnetic field of Helmholtz coils is given by

\begin{equation}
		B = \left(\frac{4}{5} \right) ^{3/2} \frac{\mu_0 nI}{R}.
		\label{eq:hcoils}
\end{equation}

We want to have the following conditions satisfied:

\begin{itemize}
		\item $I = \SI{1}{A}$
		\item $\SI{0}{g} \leq B \leq \SI{10}{g}$
		\item $R = \SI{27/2}{\centi \meter} = \SI{.135}{\meter}$
\end{itemize}


Solving for the number of turns in each Helmholtz coil $n$ to create the maximum magnetic field of $\SI{10}{gauss}$, we find

\begin{equation}
\begin{align*}
		n & = \left( \frac{5}{4} \right) ^{3/2} \frac{RB}{\mu_0 I}\\
		  & = \left( \frac{5}{4} \right) ^{3/2} \frac{(\SI{.135}{\meter}) (\SI{.001}{T})}{\left( \SI{4\pi e-7}{T \meter \per A}\right) \left( \SI{1}{A} \right) }\\
		  & \approx \SI{150}{turns}
\end{align*}
		\label{ncoils}
\end{equation}


With of wire gauge of about 20 AWG (is this about standard?), the wire will have a diameter of about $w_d = \SI{1}{\milli \meter}$ and will be able to hold up to $\SI{10}{A}$. The width of the coil will then need to be

\begin{equation}
\begin{align*}
				L_{coil} &= \left( \frac{n}{x} \right) w_d\\
						 & = \left( \frac{150}{3} \right) \SI{1}{\milli \meter}\\
				& = \SI{5}{\centi \meter}
		\end{align*}
		\label{length of coil}
\end{equation}

where $n$ is the number of turns and $x$ is the number of times the wire is layered (3 layers in this situation).


We will thus need length of wire 
\begin{equation}
		\begin{align*}
				l_{wire} & = n * \text{circumference} * \SI{2}{coils}\\
			 & = \SI{150}{turns}*  2\pi \left( \SI{27/2}{\centi \meter} \right) * 2\\
				& = \SI{250}{\meter}
		\end{align*}
		\label{length of wire}
\end{equation}



\end{document}
