\documentclass[]{article}

\begin{document}
Returning to the idea of telescopes, light that is emitted from a distant star or reflected off a moon or planet will essentially travel through free space for the majority of its voyage. However, as it nears Earth, it begins to enter the atmosphere that surrounds Earth, and refracts according to Snell's Law. It would seem that this refraction is the same for all rays of light, and thus does not result in any imperfections when imaging with a telescope. The problem stems from two issues: The first issue is that the index of refraction depends on many parameters such as temperature, pressure, and density. The second issue is that the atmosphere of the Earth is not constant in these parameters, but fluctuates slightly over time. Putting this two issues together tells us that light will not refract uniformly as it passes through the atmosphere, but will be distorted due to fluctuations in the index of refraction. This is known as atmospheric distortion.

\end{document}
