Atoms with a magnetic moment experience a torque in an external magnetic field, leading to a precession of the atom's total angular momentum vector about the axis of the external magnetic field (Larmor Precession). This precession degrades the benefits obtained through optical pumping, a technique used to increase atomic fluorescence, through a redistribution of the atom's angular momentum. However, a technique known as \textit{magnetic resonant pulsing}\footnote{Kane, Thomas J., Paul D. Hillman, and Craig A. Denman. ``Pulsed laser architecture for enhancing backscatter from sodium.'' Proc. of SPIE. Vol. 9148. 2014.} suggests that laser light pulsed at the corresponding Larmor frequency can mitigate this degradation and increase atomic fluorescence by reestablishing the benefits of optical pumping. We test this technique by constructing a pulsed laser and measuring the fluorescence of rubidium atoms confined in various magnetic field configurations. Magnetic resonant pulsing has applications in optical magnetometry, atomic fluorescence, and laser guide stars.

