State-of-the-art telescope systems are constantly being improved so that they can see celestial objects with greater clarity. One way telescopes achieve greater clarity is by using a combination of a laser guide star (LGS), an artificial star created by shining a laser into the upper atmosphere, where the laser light is absorped and emitted by sodium atoms, along with image processing algorithms. The brightness of the LGS, however, is diminished by the torque the geomagnetic field exerts on the magnetic moment of the sodium atoms. This is due to the precession that the atoms undergoes after experiencing the magnetic torque, causing a change in the atom's total angular momentum with time, decreasing the absorption and emission of light by the atom. However, with a theoretical process known as \textit{magnetic resonant pulsing}\footnote{Kane, Thomas J., Paul D. Hillman, and Craig A. Denman. ``Pulsed laser architecture for enhancing backscatter from sodium.'' Proc. of SPIE. Vol. 9148. 2014.} (MRP), if the incoming laser light is pulsed with a repetition rate equal to that of the atom's precession frequency, known as the Larmor frequency, the negative effect of the magnetic field can be mitigated and the brightness of the LGS can be increased. We experimentally test MRP by confining rubidium atoms in a magnetic field and measure the absorption of laser light pulsed at the corresponding Larmor frequency.
