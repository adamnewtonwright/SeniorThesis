\documentclass[]{article}
\usepackage{graphicx}
\linespread{1}
\usepackage{amsmath}
\usepackage{siunitx}
\usepackage{tikz}
\usepackage[]{geometry}
\usepackage{pdfpages}
\title{Increasing Laser Guide Star Fluorescence through Magnetic Resonant Pulsing}
\author{Adam Wright}
\begin{document}
\maketitle
%Atoms with a magnetic moment experience a torque in a magnetic field, leading to a precession of the atom's total angular momentum about the axis of the magnetic field (Larmor precession). This precession degrades the benefits obtained through optical pumping, a technique used to increase fluorescence, through a redistribution of the atom's angular momentum. However, a technique called \textit{magnetic resonant pulsing} (Kane et al., 2014.) suggests that light pulsed at the Larmor frequency can mitigate this degradation and increase atomic fluorescence. We test this technique by constructing a pulsed laser and measuring the fluorescence of rubidium atoms in a magnetic field.

Laser guide stars (LGS) are an essential tool in modern astronomy, used to correct and restore distorted images of celestial objects. One important characteristic of a LGS's performance is its brightness. However, due to the geomagnetic field causing a precession of the atom's total angular momentum vector (Larmor precession), the brightness of a LGS is significantly diminished. By constructing a pulsed laser and magnetic field housing, we measure the fluorescence of rubidium atoms in order to test a technique known as \textit{magnetic resonant pulsing} (Kane et al., 2014) that can mitigate the effects of Larmor precession and increase LGS brightness.
\end{document}
