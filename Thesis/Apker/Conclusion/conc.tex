\chapter{Conclusion}
In this thesis, we introduce laser guide stars and describe an inherent problem that hampers their efficiency: Larmor precession. We thoroughly explain the theory behind laser guides stars and the theory behind the problem of Larmor precession that degrades the brightness of laser guide stars. From this, we identify a possible solution to increase the brightness of laser guide stars: magnetic resonant pulsing. The theory of magnetic resonant pulsing is explained and an experiment to test magnetic resonant pulsing is presented and demonstrated.

We confirm that magnetic resonant pulsing indeed results in an increase in fluorescence. For circularly polarized light and a pulsed laser beam, we report a 14\% increase in fluorescence between repetition rates that are out of and in resonance with the Larmor precession. Furthermore, we show that for circularly polarized light, magnetic resonant pulsed light decreases in fluorescence by only 2\% when the angle between the magnetic field and laser beam increases from $0^{\circ}$ to $90^{\circ}$. In contrast, a continuous wave laser decreases by 32\% when the angle between the laser beam and magnetic field increases from $0^{\circ}$ to $90^{\circ}$.

These data show that laser guide star systems at most latitudes would benefit from magnetic resonant pulsed lasers. A significant challenge with this, however, would be matching the repetition rate of the laser to the Larmor frequency of the atom. The precise value of the geomagnetic field is not always known, and thus a laser system capable of variable repetition rate would be needed in order to lock to magnetic resonant pulsing. 

However, there is still much more work that can be done in this area. One significant problem encountered in this experiment was measuring the fluorescence while changing the magnetic field strength, orientation of the magnetic field, or the repetition rate. One of these parameters would be changed, and a measurement would be recorded manually from the oscilloscope. This took around \SI{30}{\second} per measurement and for approximately 20 measurements, this time would add up. In this time, it was possible for the diode laser to stray off resonance, affecting the fluorescence measurements significantly. For future experiments, this should be taken into consideration. Possible solutions would be a more stable laser source or a computerized system to instantly record values from the photodiode. We considered an Arduino system that would read analog voltages at sub-millisecond intervals to a computer, but did not have time to implement this.

Another problem was generating a precise magnetic field and compensating for the geomagnetic field. For studies looking at this effect with greater precision, it would be beneficial to use a magnetic field shield in order to rid the system of all stray magnetic fields. A shield using MuMetal was considered but turned out to be too costly. This would allow for greater accuracy in the angle between the magnetic field and laser beam. This could also be done using a system of three Helmholtz coils along the three Cartesian axes.

An area that should be explored in more detail is that of the duty cycle of the laser. Our laser had a minimum duty cycle of 20\%. This meant that laser light was interacting with the atoms for 20\% of their Larmor precession. In this time, a significant change in the atoms' total atomic angular momentum vector is occurring, decreasing the benefits of optical pumping. A duty cycle of 20\% is found numerically to be optimal \cite{Rampy2010}, but lower duty cycles could result in further increases the benefits of magnetic resonant pulsing, and should be studied further.

Furthermore, the transition saturation problem should also be further investigated as this is not completely independent from Larmor precession. Transition saturation occurs when a significant number of atoms transition into the lower ground state (F=1) of rubidium, becoming inaccessible to excitation from laser light. Adding the repumper at different intensities will shed more light onto this problem. In addition, the polarization and repetition rate of the repumper could be varied.

Another area to look at is the fluorescence versus power of the laser while in magnetic resonant pulsing. Performance of laser guide stars with respect to power have been studied \cite{Holzlohner2012}, but not in magnetic resonant pulsed systems, and it would be beneficial to study this area and look into saturation intensities. The spot size of the laser guide star is another area that has been studied, \cite{Holzlohner2012} and could be studied in this system by varying the width of the laser beam.
