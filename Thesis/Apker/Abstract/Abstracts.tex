\begin{abstract}

%\section*{General Abstract}
%
%\noindent Modern astronomy relies on the ability to take extremely high resolution images of distant celestial objects, such as stars and galaxies. However, in order to obtain such high resolution images, astronomers must use a combination of laser guide stars --- artificial stars created by shining laser light into the upper atmosphere where it is reflected back to earth by atoms --- and computer algorithms to restore distorted images. In this thesis, we look at improving the brightness of laser guide stars by using a laser that emits bursts of light instead of a continuous beam of light. We build this proposed laser, model magnetic field conditions, and measure the brightness of our experimental laser guide star, showing that pulsed lasers indeed create brighter laser guide stars, especially for locations near the equator where the geomagnetic field is perpendicular to the laser beam.
%
%
%\section*{Technical Abstract}

\noindent Atoms with a magnetic moment residing in a magnetic field experience Larmor precession, a precession of the atom's total atomic angular momentum due to the torque the magnetic field exerts on the atom, with frequency known as the Larmor frequency. This precession degrades the benefits obtained through optical pumping, a technique used to increase atomic fluorescence, through a nonoptimal redistribution of the atom's angular momentum. However, according to Kane et al. (2014), a technique known as magnetic resonant pulsing suggests that laser light pulsed at the Larmor frequency can mitigate this degradation and increase atomic fluorescence by reestablishing the benefits of optical pumping. We test this technique by constructing a pulsed laser and measuring the fluorescence of rubidium confined in a magnetic field. We report a 32\% increase in fluorescence between CW and pulsed laser light when the laser beam is perpendicular to the magnetic field. This method can significantly increase the irradiance of laser guide stars due to the near-equatorial position of many telescopes, giving angles between their laser beams and the geomagnetic field of nearly $90 ^{\circ}$.

\end{abstract}
